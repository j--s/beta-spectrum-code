\documentclass{report}
\usepackage{tikz}
\usetikzlibrary{shapes.geometric, arrows}
\usepackage{booktabs}
\usepackage{graphicx}
\usepackage{float}
\usepackage{caption}
\usepackage{multirow}
\usepackage{subcaption}
\usepackage{amsmath}
\usepackage[margin=1.0in]{geometry}


\begin{document}


\title{Beta Spectrum Code Documentation}
\author{Department of Physics, Temple University\\ Jonathon Sensenig}
\maketitle
\chapter{Theory}
\section{Introduction}
\paragraph{}
The beta spectrum analysis code was written with the goal of calculating the beta spectrum
resulting from the spontaneous fission of $^{252}$Cf. Since the beta spectrum of this element
is not well known it is of general interest to measure it experimentally. The design of this
experiment will be guided by the beta spectrum calculations. The caveat is the beta decay 
data which is tabulated in nuclear databases such as ENSDF, JEFF, etc. is not completely 
reliable. This uncertainty must be kept in mind while generating beta decay spectra. 
\section{$^{252}$Cf}
\paragraph{}
The element $^{252}$Cf is one of the heaviest elements on the periodic table. Due to its large
mass it decays with either an alpha decay or fission, with a 97$\%$ and 3$\%$ probability, respectively.

\begin{figure}[H]
    \centering
        \includegraphics[width=100mm]{cf-252-cumfiss-plot.png}
        \caption{Plot of the cumulative fission yield of $^{252}$Cf versus the atomic number $A$. 
The cumulative fission yield includes the beta decays of the fission fragments as well as the
 subsequent beta decay chains.}
\end{figure}


\section{Beta Decay}
\paragraph{}
Beta decay comes in two varieties, plus and minus. The plus decay is a result of the decay of 
a proton to a neutron and the minus decay a neutron decaying to a proton.
\begin{center}
Beta Plus:      $p \rightarrow n + e^+ + \nu_e$\\
Beta Minus: $n \rightarrow p + e^- + \bar\nu_e$
\end{center}
in terms of the nuclei the following is the generic decay from parent to daughter nucleus,
\begin{center}
Beta Plus: $ ^{Z} X^A \rightarrow ^{Z-1}X^{A+1} + e^+ + \nu_e$ \\
Beta Minus: $ ^{Z} X^A \rightarrow ^{Z+1}X^{A-1} + e^- + \bar\nu_e$
\end{center}
Nuclei beta decay to reach a more stable state, usually a nucleus with lower mass. 
The bound nucleus is a struggle between the Coulomb repulsion, between the protons and the attraction
between the nucleons (protons and neutrons), and the much stronger, strong nuclear force. The strong
 nuclear force also depends on the spin of the nucleons. A nonzero nucleon spin results in a stronger 
 attraction compared to zero spin. 
The strong nuclear force favors an even number of proton and neutrons. Since protons and 
neutrons are not identical particles they can be in the same spin state without violating the Pauli
exclusion principle. Thus nuclei favor and decay to these "magic numbers" (even Z and A), through beta decay, 
since they are more strongly bound and thus lower energy. In the end the strong nuclear force "wins" the struggle,
 binding the nucleons at very short distances, 
since it is very strong over a short range. It loses most of its effectiveness over distances greater than 2.5 Fermi. 

\subsection{Decay Probability}
\paragraph{}
The spectrum of the beta decay with energy endpoint $E_0 = E_{\nu} + E_e$ where $E_0$ is 
the endpoint energy for the decay and sometimes called the $Q$ value. The probability 
that a decay will occur for a given electron energy $E_e$ is,
\begin{equation}
\frac{dN}{dE_e}(E_e, Z, A) = N_0(E_e) F(E_e, Z, A) C(E_e) (1 + \delta(E_e, Z, A))
\end{equation}
Here $N_0$ is the electron momentum, F is the Fermi function, C the shape factor, and
 $\delta$ is the sum of three corrections (see subsection 1.3.x). The electron's momentum is given by,
\begin{equation}
N_0(E_e) = (G_F^2/2\pi^3)\left[p_e E_e (E_0 - E_e)^2\right]
\end{equation}
The Fermi coupling constant is $G_F$, $p_e$ is the electron momentum, and $E_{\nu} = (E_0 - E_e)$ is 
the neutrino momentum. Ultimately the factor of $G_F/2\pi^3$ is absorbed into the normalization constant
so it does not enter the calculation.

It can be useful (it is what is used in the code) to convert the electron energy to kinetic energy with the simple relation,
\begin{equation}
E_e = T_e + m_e
\end{equation}

\subsection{The Fermi Function}
\paragraph{} 
The Fermi function accounts for the Coulomb potential felt by the emitted electron. For the beta minus 
decay the electron will feel an attraction to the nucleus from the Z protons. 

 The Fermi function can be broken into two parts, $F_0$ and $L_0$. This allows for an analytic form
 for $F_0$ and approximations to be made for $L_0$. The approximation $L_0 = \frac{1}{2}(1+\gamma)$ is used here. Then the form of the Fermi function $F$ is,
\begin{equation}
F(Z,E) = L_0F_0 = L_0 \left[4(2p_eR)^{-2(1-\gamma)}\frac{|\Gamma(\gamma+iy)|^2}{[\Gamma(2\gamma+1)]^2}e^{\pi y}\right]
\end{equation}
where $p_e = \sqrt{T_e^2 + 2m_eT_e}$  where $T_e$ is the electron kinetic energy.\\ \\
$R = r_0 A^{1/3}$ where $r_0 = 0.0031$ (Ref. 5) in natural units (c = $\hbar$ = 1). \\ \\ 
$\gamma = \sqrt{1 - (\alpha Z)^2}$  where $\alpha$ is the fine structure constant.\\  \\
$y = \alpha Z E_e / p_e$ where $p_e$ and $E_e$ are defined above.\\ \\ 
$\Gamma(x)$ is the gamma function.\\

An approximation of the Fermi function can be made which is described in Ref.(3). This approximation is simple
 and accurate for most of the neutrino momentum range and nuclei masses. However, the 
anti-neutrino momentum is the inverse of the electron momentum. Thus when calculating the beta spectrum it reaches 
the accuracy limits of the approximation. This is one reason a more accurate approximation 
is desirable, especially for larger mass nuclei and very high or low electron momenta.
\begin{equation}
F(Z, E) = \frac{E}{p}exp\left[ \alpha(Z) + \beta(Z) \left( \frac{E}{m_e c^2} -1 \right)^{1/2} \right]
\end{equation} 
The functions $\alpha(Z)$ and $\beta(Z)$ are fitted to experimental data. The functions split the momentum into 
two ranges, the first being for electron kinetic energy, 
$T < 1.2m_e c^2$
\begin{equation}
\alpha(Z) = -0.811 + 4.46  \times 10^{-2} Z + 1.08 \times 10^{-4} Z^2 
\end{equation}
\begin{center}
$\beta(Z) = 0.673 - 1.82 \times 10 ^{-2} Z + 6.38 \times 10^{-5} Z^2$
 \end{center}
 The second range, $T \geq 1.2m_e c^2$, is covered by the similar functions,
\begin{equation}
\alpha(Z) = -8.46 \times 10^{-2} + 2.48  \times 10^{-2} Z + 2.37 \times 10^{-4} Z^2 
\end{equation}
\begin{center}
$\beta(Z) = 1.15 \times 10^{-2} + 3.58 \times 10 ^{-4} Z - 6.17 \times 10^{-5} Z^2$
 \end{center}

\subsection{Shape Factors}
The shape factors $C$ have multiple expressions which are assigned according to the parity and spin change, 
$\pi_i\pi_f$ and $\Delta J$, respectively.
\begin{table}[H]
\centering
\caption{Selection rules for beta decay. The forbiddeness is the amount that a particular decay 
is suppressed. The suppression of the decay goes from least suppressed at the top of the table to
 most suppressed at the bottom. Here J and $\pi$ are the nucleus' spin and parity, respectively. (6)}
\renewcommand{\arraystretch}{1.75}
  \begin{tabular}{lllll}
    \toprule 
      $ \Delta $J  = $|J_f - J_i|$& & $\pi_i \pi_f$ &  & Classification  \\
       \midrule
  	0, 1 & &1 &  & Allowed  \\
   	 0, 1 & &-1 &  & 1$^{st}$ Forbidden \\
	  $>$ 1 & &(-1)$^{|\Delta J|}$ &  & $|\Delta J|^{th}$ Forbidden \\
	 $>$ 1 & &(-1)$^{|\Delta J| - 1}$ &  & $(|\Delta J| - 1 )^{th}$ Forbidden Unique  \\
    	   \bottomrule
  \end{tabular}
\end{table}
The form of the shape factors are given in (2) with a longer discussion of the shape factor effects on 
the beta and anti-neutrino spectrum in (1). The factors are tabulated according to forbiddeness  and the spin 
of the emitted electron and anti-neutrino. If the spin of the electron and anti-neutrino are anti-parallel, their spins
cancel, since they are both spin 1/2, such that S = 0. This spin pairing is given the name Fermi decay. Conversely, if the spins couple 
such that S = 1, it is called a Gamow-Teller decay. 

Below is a list of the shape factors implemented in the code. The Gamow-Teller factors were chosen 
since it includes a factor for the 1$^{st}$ forbidden unique decay. Choosing the 0$^-$ over the 
1$^-$, for the 1$^{st}$ forbidden was somewhat arbitrary. 
\begin{table}[H]
\centering
\caption{The Gamow-Teller shape factors from (2). $\beta = p_e/ E_e$}
\renewcommand{\arraystretch}{1.75}
  \begin{tabular}{llll}
    \toprule 
      Classification & $\Delta J^{\pi}$ & Shape Factor C($E_e$) &    \\
       \midrule
  	Allowed &1$^+$ & 1 &     \\
	1$^{st}$ F. &0$^-$ & $p_e^2 + E_{\nu}^2 + 2\beta^2E_eE_{\nu}$ &     \\
	1$^{st}$ F. &1$^-$ & $p_e^2 + E_{\nu}^2 - \frac{4}{3}\beta^2E_eE_{\nu}$ &     \\
	1$^{st}$ F. Unique &2$^-$ & $p_e^2 + E_{\nu}^2$ &     \\
    	   \bottomrule
  \end{tabular}
\end{table}
 
\subsection{Corrections}
 The $\delta$ correction Eqn.(1.1) is the sum of three corrections $\delta = \delta_{Rad} + \delta_{FS} + \delta_{WM}$. These three
corrections are $\delta_{Rad}$ the radiative, $\delta_{FS}$ the finite size, and $\delta_{WM}$
the weak magnetism corrections. Expressions for the $\delta_{FS}$ and $\delta_{WM}$ corrections can be found in (1) and (2), although the weak magnetism corrections are somewhat complicated. 

The finite size (FS) correction removes the approximation that the nucleus is a point. It is also decay
dependent and inexact. An approximation is given in (1) for allowed Gamow-Teller decays.
\begin{equation}
\delta_{FS} = - \frac{3}{2} \frac{\alpha Z}{\hbar c} \langle r \rangle _{(2)} \left( E_e - \frac{E_{\nu}}{27} + \frac{m_e^2 c^4}{3E_e}\right)
\end{equation}
where, assuming uniform distributions of weak and charge densities of radius R, $\langle r \rangle _{(2)} = (36/35) R$. 
A more complete description of this correction is given in (1).

The radiative correction $\delta_{rad}$ is derived for allowed decays in the first order of $\alpha$ in (7).
The function is listed as $g(E_e, E_0, m_e)$ or Eqn.(20b), where the notation is consistent with what is used
above.

\chapter{Code}

\section{Introduction}
\paragraph{} 
This beta spectrum code consists of two main scripts, one to extract the values from the ENSDF files and
 one to create the full beta spectrum.
The dataset processing script is separated from the spectrum generating code to allow easy adaptation for different data formats from various databases. 
 Only the scripts designed to extract the data from the files need to be written or modified, while 
 the script generating the spectrum knows nothing of the change.
 
 The following instructions are based on my experience using ENSDF files. The experience using other databases might be slightly different.
 \section{Data Flow}
 \paragraph{}
 The data flow is relatively simple. Start by downloading the entire beta decay dataset from the 
 ENSDF website. The dataset will be in a zip file so extract it to a convenient directory. Place the 
 two python scripts, \textit{read$\_$ensdf.py} and \textit{isotope$\_$spectrum.py}, and the fission yield
 file \textit{cum$\_$py.txt} in the same directory. 
 Now you're ready to create your spectrum! Below is a diagram of the required files and processing 
 scripts along with the files the scripts create.
 \\ \\ 
 \tikzstyle{startstop} = [rectangle, rounded corners, minimum width=3.2cm, minimum height=1cm,text centered, draw=black]
\tikzstyle{io} = [rectangle, rounded corners, minimum width=3.2cm, minimum height=1cm, text centered, draw=black]%, fill=blue!30]
\tikzstyle{process} = [rectangle, minimum width=4cm, minimum height=2cm, text centered, text width=3cm, draw=black]%, fill=orange!30]
\tikzstyle{decision} = [diamond, minimum width=3cm, minimum height=1cm, text centered, draw=black, fill=green!30]
\tikzstyle{arrow} = [thick,->,>=stealth]

\begin{tikzpicture}[node distance=2cm]

\node (in1) [io] {cum$\_$py.txt};
\node (in2) [io, below of=in1] {ENSDF Files};
\node (pro1) [process, right of=in1, xshift=4cm, yshift=-1cm] {read$\_$ensdf.py \\ Data Extraction Code};
\node (out1) [io, right of=pro1, xshift=3.3cm, yshift=1cm] {betadata.dat};
\node (out2) [io, right of=pro1, xshift=3.3cm, yshift=-1cm] {readerr.dat};

\draw [arrow] (in1) |- (pro1);
\draw [arrow] (in2) |- (pro1);
\draw [arrow] (pro1) -| (out1);
\draw [arrow] (pro1) -| (out2);
\end{tikzpicture}
\\ \\ \\
\begin{tikzpicture}[node distance=2cm]

\node (in1) [io,  xshift=1.0cm] {cum$\_$py.txt};
\node (in2) [io, below of=in1] {betadata.dat };
\node (pro1) [process, right of=in1, xshift=4.5cm, yshift=-1cm] {isotope$\_$spectrum.py \\ Beta Spectrum Generating Code};
\node (out1) [io, right of=pro1, xshift=3.5cm, yshift=1cm] {final$\_$spectrum.dat};
\node (out2) [io, right of=pro1, xshift=3.5cm, yshift=-1cm] {proc$\_$results.dat};

\draw [arrow] (in1) |- (pro1);
\draw [arrow] (in2) |- (pro1);
\draw [arrow] (pro1) -| (out1);
\draw [arrow] (pro1) -| (out2);

\end{tikzpicture}

\section{Requirements}
Need Python 3.xx although it should be backward compatible with 2.xx without any problems. Also need Scipy and 
Numpy. Highly recommend matplotlib to plot the results although an alternative can be used.

\section{Scripts}
\subsection{ENSDF Processing Script (read$\_$ensdf2.py)}
\paragraph{}
The ENSDF processing script takes care of extracting the beta decay data from the ENSDF files and 
writes it to a file. The file is kept separate so that the this script can be modified or replaced without affecting the beta 
spectrum generating code. This allows various databases and formats to be used as the source
of the decay data. 

This script currently reads from the cumulative fission yield file (\textit{cum$\_$py.txt}), extracting the
 parent nuclei, isomer flag, and fission yield. The script then searches the ENSDF files in the current directory, reading the 
data from the matching files and storing the data in the \textit{betadata.dat} file and any errors in an error file named \textit{readerr.dat}. 

There are a number of assumptions that have to be made when reading the ENSDF files. Below is a list of
these assumptions. \\ \\
1. No branching ratio: The branch with an unlisted ratio gets assigned 0. Then all branching ratios for a 
decay are summed and the difference between this total and 100 is equally distributed to the branches with unlisted ratios. 
E.g. With 5 unlisted branches 100 - 5$\times$0 = 100 so each branch gets a ratio of 100/5. \\ \\
2. Uncertain or Questionable Decay: These decays are assigned a forbiddeness of A. \\ \\
3. Predicted or Expected Decay: These decays are assigned a forbiddeness of A. \\ \\
4. An L but no B record: Discard the daughter energy and skip the branch.\\ \\\
5. No normalization: Assume the normalization is 1.0. \\ \\
6. No daughter energy: Skip the branch. \\ \\
7. No listed parent energy: Assume ground state, i.e., E = 0. \\ \\
8. Daughter energy is X+E or E+X: Remove the X, + and treat as an energy. \\ \\

\subsection{Extracted Decay Data File (betadata.dat)}

The format of the \textit{betadata.dat} file is,

\begin{table}[H]
\centering
\renewcommand{\arraystretch}{1.75}
  \begin{tabular}{lllll}
    \toprule 
     Line & & Column 1 & Column 2  & Column 3  \\
       \midrule
  	1. & &Parent Nuclei (string) &  Isomer Tag (0,1)&  \\  	
	2. & &Parent Energy (float)&   Q Value (float)  &\\  
	3. & &Daughter Energy (float)& Forbiddeness (X)&  Branch Ratio (float)  \\  	
	$\vdots$ & &  &  &     \\  
	N. & & EOF &  &     \\  			
	\midrule
	X Values  &&&&\\
	\midrule
         A (Allowed), &&&&\\
	FF (1$^{st}$ forbidden) &&&&\\
	FFU (1$^{st}$ forbidden unique), &&&&\\
	SFU (2$^{nd}$ forbidden unique)&&&&\\
    	   \bottomrule
  \end{tabular}
\end{table}


%1. $\hspace{10mm}$ Parent Nuclei 	$\hspace{23mm}$	Isomer tag (0 / 1)\\
%2. $\hspace{10mm}$ Parent energy level 	$\hspace{15mm}$	Q - value\\
%3. $\hspace{10mm}$ Daughter energy level $\hspace{10mm}$	Forbiddeness 	$\hspace{15mm}$Branch Ratio \\
 %$\vdots$ \\
%4. $\hspace{10mm}$ EOF

1. The first line is the parent nuclei, note the ENSDF file is named after the daughter nuclei, e.g., parent 16N with the daughter
 (and ENSDF file name) being 16O. The Isomer flag is 0 if it is a ground state decay and 1 if
the decay is from an excited state (isomer).\\ \\
2. This line gives the parent energy, 0 (ground state) unless it is an isomer. The second column is the Q value, or endpoint 
energy, of the decay.\\ \\
3. The first column is the daughter energy level it is decaying to. The second is the Forbiddeness of 
the decay, with the possible values listed under \textit{X Values}, above. The third column is the 
branching ratio, or probability of the given decay branch occurring.  \\ \\
4. Indicates the end of the parent or isomer decay data. 

\subsection{Error File (readerr.dat)}

The script \textit{read$\_$ensdf2.dat} also prints an error file called \textit{readerr.dat}. This lists the 
daughter nuclei/ENSDF file name and if the file exists. If the file exists it lists the branching ratios and
the total branching ratio (sum of all the ratios). If this value is greater than 100$\%$ it lists an error. An example
 of an entry in \textit{readerr.dat} with a few errors is listed below.\\
------------------------------------------\\
------------------------------------------ \\
File:  103NB \\
No Branching Ratio on line:  37\\
No Branching Ratio on line:  56\\
No Branching Ratio on line:  80\\
No Branching Ratio on line:  115\\
----------- \\
Branching Ratios \\
10.0\\
0.0\\
0.0\\
0.0\\
0.0\\
Total Branching Ratio:  10.00\\
-----------\\
File:  103NB\\
Length mismatch!\\
Length of e:  6 :  [0.0, 163.9, 367.5, 620.0, 1472.3, 2387.7]\\
Length of intensity:  5 :  [10.0, 0.0, 0.0, 0.0, 0.0]\\
Length of forbidden:  5 :  ['A', 'A', 'A', 'A', 'A']\\
Successfully Read File\\
------------------------------------------\\
------------------------------------------\\
There are no branching ratios listed for 4/5 branches listed. The error message is followed by the line
 in the ENSDF file where the error occurred. These 4 branches are assigned a ratio of 0 as stated in Sec. (2.4.1). 
 The last error is the length mismatch. It prints the entire list after the length for easier diagnostics.
  For every energy there should be an intensity and forbiddeness so the
number of elements in the three lists should match. However, if the last daughter energy is not followed 
by a line with a branching ratio and forbiddeness, the e list ends up with an extra element. Since 
a daughter energy with no branching ratio and forbiddness is skipped in anyway, this is not a problem but
one should be aware of this if modifications are made. 

\subsection{Beta Spectrum Generator (isotope$\_$spectrum.py)}
\paragraph{}
This script is where the beta spectrum is generated. It is probably most intuitive to list its operation as a list of steps. \\ \\
1. The parent nucleus and its fission yield is read from the \textit{cum$\_$py.txt} file.\\ \\
2. The respective nuclei's decay data is retrieved from the \textit{betadata.dat} file, if it exists.\\ \\
3. If the data is found in step 2, a beta spectrum is generated for each branch. Each spectrum is
 normalized, such that the integrating over the spectrum gives 1,  then multiplied with the given branch ratio.\\ \\
4. The spectra from each branch of a isotope or isomer are summed. The total is normalized and multiplied
by the fission yield. \\ \\
5. The spectra from every isotope and isomer is summed and normalized. \\ 

The script writes the final spectrum to \textit{final$\_$spectrum.dat}. This file contains two columns, the first is
the energy and second the decay probability. 

\subsection{Beta Spectrum Script Calculation Details}
There are two versions of the beta spectrum code. The \textit{isotope$\_$spectrum.py} uses the 
approximation for the Fermi function which is given in Eqn.(1.5) and Eqn.(1.6). This is the most 
straightforward implementation and performs well for most nuclei. However, it approaches its limits 
for large nuclei such as some of the  $^{252}$Cf fission fragments. A full discussion of the limits
of the approximation is given in (3). This leaves some room for improvement along with the 
question of how the implementation of an exact Fermi function would perform.

The script \textit{isotope$\_$spectrum$\_$exfermi.py} is the same except for the exact Fermi function Eqn.(1.3) 
and the addition of a finite size ($\delta_{FS}$) correction Eqn.(1.7). 

Note both beta spectrum generating codes use only the Gamow-Teller shape factors. The code 
only uses the 0$^-$ 1$^{st}$ forbidden shape factor. See section 1.3.3 for more details.

\subsection{Fission Yield (cum$\_$py.txt)}
This file contains the cumulative fission yield for the $^{252}$Cf spontaneous fission. The format is from the 
human readable format on the ENDF website at, 
\begin{center}
 http://www.nndc.bnl.gov/sigma/getFissionYieldsData.jsp?evalid=15509$\&$mf=8$\&$mt=459 
 \end{center}
 Not sure where the nucleus names came from. Each line of the file is an isotope with the columns 
 in the following format,
 
 \begin{table}[H]
\centering
\renewcommand{\arraystretch}{1.75}
  \begin{tabular}{ll}
    \toprule 
     Column & Description\\
       \midrule
  	 1. &Line Number\\
	  2.&Z \\
	   3.& Parent Nucleus Name\\
	    4.& A \\
	    5. & ?? \\
	     6.& Isomer Tag \\ 
	      7.& Fission Yield  \\
	       8.& Uncertainty in Fission Yield \\
    	   \bottomrule
  \end{tabular}
\end{table}
\section{Beta Spectrum Processing Results (proc$\_$results.dat)}
\paragraph{} 
This file contains information regarding the processing of the beta decay data. For each parent 
nuclei, it contains the number of branches, and their respective energy levels, forbiddeness, and 
branching ratio. It also gives the normalization integration, for each level and isotope or isomer. The 
file's content labelling is self explanatory and it should only be necessary to use for diagnostic purposes. 

 
\chapter{Hopes, Dreams, and Reality}
As an order of priority, the following items should be addressed in the order which they are listed.
\section{Fixes}
\begin{figure}[H]
    \centering
        \includegraphics[width=100mm]{current_spec_plot.png}
        \caption{My current code overlaid on Gabe's spectrum. The oval indicates the area of disagreement.}
\end{figure}

The code is currently in a state where it can generate a spectrum which appears to be mostly correct. Gabe 
has developed code which can also generate beta spectra. Using his spectrum as a check, most of the 
spectrum matches nicely except in the  10,500 - 12,500 keV range, as seen in Fig.(3.1). I'm not sure 
which spectrum is correct but below is a list of possible reasons for the discrepancy. \\ \\ 
1. Handling decays which do not have listed branching ratios. \\ \\
2. Handling decays which do not have a branching ratio and forbiddeness (no B line in ENSDF file).\\ \\
3. The daughter energies with +X added. see section 2.4.1 assumption 8. \\ \\
4. The shape factors used for the forbiddeness could be different.\\ \\
5. I use natural units, i.e., $\hbar = c =1$. Not sure if this is right so it's worth checking. \\ \\
I would recommend resolving this discrepancy as the first priority, in that if this is not understood, it's hard to say if the spectrum is correct.

\section{Upgrades}
This section focuses on developing the exact Fermi function and adding corrections. It is the most 
physics-y part of the project and therefore the most interesting, although, arguably, there are no boring parts. 
The following is a list of enhancements that can be made to the code.   \\ \\
1. The radiative correction should be added. There is a complicated form given in (7), which is 
only for allowed decays but should make around 1$\%$ difference. \\ \\
2. The weak magnetism corrections as given in (1) and (2).  \\ \\
3. Find a more accurate form for the $L_0$ function in the Fermi function. Currently, it is given by
$L_0 = \frac{1}{2}(1+\gamma)$ but there is a derivation of a more accurate approximation in (5), albeit 
much more complicated. \\ \\
4. More accurate assignment of the shape factors, specifically choosing between the 
Gamow-Teller and Fermi decay forms. This requires a bit of ingenuity in calculating the
 spin and parity change $\Delta$J and $\pi_f\pi_i$, respectively. Quite often there are multiple $J^{\pi}$
 values given for each branch, in the ENSDF file so one would need to develop some standard 
 for choosing a single value for a given branch.
 

\section{Code Projects}
The script currently uses the file \textit{cum$\_$py.txt} as the input file for the fission fragment nuclei and yields. Due 
to time constraints I copied the fission yield data from the ENDF database in a human readable form. 
This works fine for now but to make the code more generic a script similar to \textit{read$\_$ensdf2.py} 
should be created to extract the fission yields from the ENDF formatted file. It could then be written to 
a file with a format similar to the \textit{cum$\_$py.txt} or some other convenient format. 

As a more futuristic idea, additional scripts could be written to read decay data from other databases 
such as JEFF.


\section{References}
1. Hayes A.C. et al. Phys. Rev. Lett 112:202501 (2014)\\
2. A. C. Hayes, Petr Vogel arXiv:1605.02047\\
3. Schenter G. and Vogel P. Nucl. Sci. Eng. 83:393 (1983)\\ 
4. Evaluated Nuclear Structure Data File, Jagdish K. Tuli (http://www.nndc.bnl.gov/nndcscr/documents/ensdf/ensdf-manual.pdf)\\
5. Behrens H. and Buhring W., Electron Radial Wave Functions and Nuclear Beta-decay (Clarendon
Press, Oxford, 1982) \\
6. Online source: https://indico.cern.ch/event/353976/contributions/1759109/attachments/701282/962803/ \\ Subatech$\_$2015$\_$mougeot.pdf  \\
7. A. Sirlin, Phys. Rev. 164, 1767 (1967); A. Sirlin, Phys. Rev. D 84, 014021 (2011). \\

%\begin{thebibliography}{9}
%\end{thebibliography}


\end{document}